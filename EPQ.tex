\documentclass[10pt,a4paper]{report}

%Packages
\usepackage[utf8]{inputenc}
\usepackage[margin=1in]{geometry}
\usepackage{amsmath,amsfonts,amssymb,amsthm}
\usepackage[english]{babel}
\usepackage{csquotes}
\usepackage{chngcntr}
\usepackage[titletoc]{appendix}
\usepackage{listings}
\usepackage{xcolor}

%Citations
\usepackage{biblatex}
\addbibresource{references.bib}

%Images and Graph Plots
\usepackage{graphicx}
\graphicspath{{images/}}
\usepackage{tikz, pgf, pgfplots}
\pgfplotsset{width=10cm,compat=newest}
\usepgfplotslibrary{external}
\tikzexternalize[prefix=tikz/]

%Code Segments
\definecolor{codegreen}{rgb}{0,0.6,0}
\definecolor{codegray}{rgb}{0.5,0.5,0.5}
\definecolor{codepurple}{rgb}{0.58,0,0.82}
\definecolor{backcolour}{rgb}{0.95,0.95,0.92}

\lstdefinestyle{code}{
	backgroundcolor=\color{backcolour},
	commentstyle=\color{codegreen},
	keywordstyle=\color{magenta},
	numberstyle=\tiny\color{codegray},
	stringstyle=\color{codepurple},
	basicstyle=\ttfamily\footnotesize,
	breakatwhitespace=false,
	breaklines=true,
	captionpos=b,
	keepspaces=true,
	numbers=left,
	numbersep=5pt,
	showspaces=false,
	showstringspaces=false,
	showtabs=false,
	tabsize=2
}
\lstset{style=code}

%Hyperlinks
\usepackage[hidelinks, pdfpagelabels]{hyperref}
\hypersetup{pageanchor=true}

\setlength{\parindent}{0mm}
\setlength{\parskip}{2mm}

%Document Information
\title{What are the limitations of derivative-based \\
	   models for optimization in machine learning?}
\author{Faris Chaudhry}
\date{\today}

%Environments
\newtheorem{theorem}{Theorem}[chapter]
\newtheorem{lemma}[theorem]{Lemma}
\newtheorem*{definition}{Definition}
\counterwithout{equation}{chapter}

%Page Numbering
\newcommand\frontmatter{
	\cleardoublepage
	\pagenumbering{roman}}
\newcommand\mainmatter{
	\cleardoublepage
	\pagenumbering{arabic}}

\begin{document}
	\frontmatter
	\maketitle

	\begin{abstract}
		Most machine learning models can be transposed into optimization problems
		with the goal being finding the global minima or maxima. This paper covers
		how the main learning methodologies (supervised, unsupervised
		and reinforcement) are essentially optimization and the mathematical models
		commonly used in the optimization along with the limitations of these methods.
	\end{abstract}

	\tableofcontents
	\mainmatter

    \chapter{Introduction to Machine Learning and Optimization}

		\section{What is Machine Learning?}
			Machine learning (ML) is a subfield of artificial intelligence (AI) which,
			broadly speaking, is the use of computational methods and models to improve
			performance and predictions through experience \autocite{FoundationsOfMachineLearning}.
			Unlike humans, this learning is based entirely on data and statistics and
			experience is gained through interaction with a training set of data
			or an environment of some kind. As a result of this,

		\section{Defining Factors of Optimization}

		\section{Notation in Optimization}

	\chapter{Supervised Learning}

	\chapter{Unsupervised Learning}

	\chapter{Reinforcement Learning}

	\chapter{Mathematical Models for Optimization}

	\printbibliography[title=References]
\end{document}
